\chapter{Introduction}

Physics currently recognizes four fundamental forces which account for nearly all known phenomena in physics.  A handful of notable observations from the past few decades, however, have identified key deficiencies in our existing model.  Much of the basic research being conducted in physics today is focused on exploring modifications or additions to the known forces and matter particles in order to provide explanations for these new observations.

This thesis describes an experimental search for such new physics using the Compact Muon Solenoid (CMS), a four-story-high detector housed 300 feet underground designed to study particle interactions produced by the Large Hadron Collider (LHC).  The LHC is the highest-energy particle accelerator ever constructed and has brought together thousands of physicists from across the globe interested in pushing the frontiers of knowledge to ever-smaller scales.

We will start by considering an overview of the Standard Model of particle physics (Chapter~\ref{chapter:standard-model}), a theory which describes the fundamental forces as arising from the exchange of various mediating \emph{bosons} between particles of matter.  The more specific focus of the thesis on $W$ and $Z$ bosons, the mediators of the weak force, is discussed in Chapter~\ref{chapter:wz-theory} along with motivation for investigations into associated $WZ$ production as a probe to reveal new physics.  Previous experimental work which informs our current understanding of weak interactions is introduced in Chapter~\ref{chap:previous}, leading to an-depth explanation in Chapter~\ref{chapter:experiment} of the capabilities of the CMS detector and the LHC.

Chapters~\ref{chapter:simulation} through~\ref{chapter:background} discuss the various tools and strategies used in building a compelling analysis of particle collision data and in particular the method used to isolate and understand a sample of $WZ$ events recorded by CMS.  All of this builds to a presentation of two new investigations performed using this collision data.  First is a measurement of the $WZ$ cross section (Chapter~\ref{chapter:cross-section}) which is a generalized description of the frequency with which $WZ$ events are produced in a particular type of collision.  Measuring that interaction probability is an important demonstration of our analysis capabilities and provides a first window for probing deviations from the predictions of the Standard Model.  In Chapter~\ref{chapter:limits}, we move on to an explicit investigation of new physics by looking for an excess of $WZ$ events clustered around a mass value corresponding to a new heavy particle.  We provide new limits on the production of such a particle and discuss the constraints they provide on a variety of proposed models for new physics.

\section{Terminology and Conventions}

In many areas of physics which involve investigations at small scales, energies are discussed not in terms of the typical SI units of Joules, but rather in terms of the electron volt (\eV), equal to the fundamental unit of charge multiplied by the SI unit of electric potential.  Most energies discussed in this text will be in terms of $\GeV = \SI{e9}{\eV}$ or $\TeV = \SI{e12}{\eV}$.

Particle interactions at the \GeV or \TeV scale are necessarily relativistic, meaning that the energies associated with the particles' rest masses ($E_0 = mc^2$) are insignificant in comparison to their kinetic energies.  Relativistic velocities are characterized by the Lorentz factor:
\begin{equation}
  \gamma = \frac{1}{\sqrt{1 - \frac{v^2}{c^2}}},
\end{equation}
with $v$ the velocity of the particle and $c$ the speed of light in a vacuum.  The total energy of a particle is given by its rest energy and momentum as $E = \sqrt{E_0^2 + (pc)^2}$ and can be expressed in terms of the Lorentz factor as $E = \gamma mc^2$, meaning that the kinetic portion of the total energy is given by $T = (\gamma - 1)mc^2$.
Considering that an electron with kinetic energy of just \sienergy{1} achieves a Lorentz factor $\gamma \approx 2000$, this relation makes clear that the rest mass of most particles plays no significant role in the relativistic limit.  As a result, we often speak of a ``\SI{10}{\GeV} electron'' or a ``\TeV muon'' where the energy value refers interchangeably to the total energy or the kinetic energy.  Indeed, particle physicists routinely drop the factors of $c$ from their equations and speak of mass and momentum in energy units, understanding that others in the community can easily infer their intended meaning.  In an effort to remain accessible to a wider audience, this thesis  maintains the distinctions between energy, momentum, and mass, along with their associated units (\GeV, \GeVc, and \GeVcc) whenever possible.

The existence of antiparticles, one of the early discoveries of the particle physics era, has become an integral piece of the field theories which describe relativistic interactions.  While antiparticles share most characteristics including mass with their particle counterparts, other properties are inverted.  When discussion demands a distinction between particles and antiparticles, it is usually sufficient to specify the electric charges; thus, an electron is designated $e^-$ while the antielectron or \emph{positron} is designated $e^+$.  In the case of neutral particles or when specification of electric charge would be distracting, an alternate notation is used where antiparticles receive an overbar; we can then distinguish a neutrino $\nu$ from an antineutrino $\bar{\nu}$ or a proton $p$ from an antiproton $\bar{p}$.  Because antiparticles are quite common in high-energy interactions, the distinction between matter and antimatter is often ignored.  Unless explicitly stated otherwise, a reference to ``electrons'' refers also to positrons while a reference to ``muons'' apply equally to $\mu^+$ as it does to $\mu^-$.

Our theoretical understanding of particle interactions is deeply mathematical, enabling us to produce incredibly precise predictions for observable processes based on various quantum field theories.  While the calculations can be complex, they can be constructed in a rather straight-forward manner from simple \emph{Feynman diagrams} (example in Fig.~\ref{fig:example-feynman}) which show the possible interactions as pictures.  For this thesis, I will use the convention that the horizontal axis of the diagram represents time, so that particles on the left represent the initial state and particles on the right represent the final state.

\begin{figure}[h]
  \centering
  \begin{tikzpicture}[thin, level/.style = {level distance = 1.6cm, line width = 1pt} ]
  \coordinate 
  child[grow = 150] { 
    edge from parent [fermion, backwards] node [above = 2pt] {$q$} 
  }
  child[grow = -150] { 
    edge from parent [fermion, forwards] node [below = 2pt] {$q^\prime$} 
  }
  child[grow = east] { 
    child[grow = 30] {
      edge from parent [boson] node [above = 2pt] {$W$}}
    child[grow = -30] { 
      edge from parent [boson] node [below = 2pt] {$\gamma$}}
    edge from parent [boson] node [above = 0pt] {$W$}
  }
  ;
\end{tikzpicture}

  \caption{An example Feynman diagram.}
  \label{fig:example-feynman}
\end{figure}

Each line represents a particle, with solid lines for fermions and wavy lines for bosons.  The arrows on the fermion lines represent the particles' momenta, meaning that arrows pointing toward the left represent particles traveling \emph{backwards} in time.  This is the Feynman diagram convention for representing antiparticles, which are indeed physically equivalent to the corresponding matter particles running in reverse.  The convention makes it easy to turn or twist the diagram to represent related processes.  By assigning momenta to the various lines and coupling values to the various vertices where those lines come together, these diagrams can be translated directly into equations which predict the probability for a given interaction.
