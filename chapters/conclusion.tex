\chapter{Conclusion}

\section{Summary}

A complete analysis of associated $WZ$ production with leptonic decays from proton-proton collisions is presented.  All investigations consider \SI{7}{\TeV} collisions produced at the LHC in 2011 recorded with the CMS detector.  Final state particles are reconstructed through software algorithms to select collision events with three well-identified, high-momentum, isolated leptons along with substantial \MET.

The $WZ$ production cross section is measured using a subset of the 2011 collision data corresponding to an integrated luminosity of \earlylumi.  A selected sample of 75 $WZ$ candidate events is compared to simulation of background events, taking into consideration the acceptance and efficiency for identifying signal events as determined from simulation.  Cross sections are determined individually for each of the four leptonic decay channels with the final result taken as the best fit linear combination, giving $\sigma(W + Z \to \ell + \nu_\ell + \ell^{\prime +} + \ell^{\prime -}) = 0.062 \pm 0.009 (\text{stat.}) \pm 0.004 (\text{syst.}) \pm 0.004 (\text{lumi.}) \pb$.

A resonance search in the $WZ$ invariant mass spectrum is performed using the full 2011 $pp$ dataset, corresponding to an integrated luminosity of \jsonlumi.  Several new particle mass hypotheses are considered, with analysis criteria optimized for each hypothesis, allowing calculation of 95\% confidence level upper limits on the cross section for a new particle in each mass window.  The cross section limits are interpreted in the Sequential Standard Model to rule out a \wprime{} with mass below \simass{1141} and in various configurations of Technicolor parameter space, greatly extending the \technirho exclusion region and disfavoring the Technicolor interpretation of CDF's dijet mass anomaly.

\section{Outlook}

Although the 2011 LHC dataset has already allowed us to reach beyond the limits set by the Tevatron on new physics in the $WZ$ channel, the results presented here are still dominated by statistical errors.  The upgrades currently in operation for the 2012 runs have driven up the center of mass collision energy by 14\% to \SI{8}{\TeV} and nearly achieved the LHC design luminosity.  The expected 2012 collision yield is four times that of the 2011 dataset, giving increased statistics for substantially more precise cross-section measurements.  The reach for a resonant search will be significantly extended by both the additional statistics and the increased collision energy.
