\chapter{The Standard Model}
\label{chapter:standard-model}


\section{History and Overview}

The Standard Model of particle physics combines into one theory all the major successes of the past century concerning our theoretical understanding of fundamental particles and their interactions.  It incorporates three of the four known fundamental forces (electromagnetism, the weak nuclear force, and the strong nuclear force), leaving only gravity out of the picture.  In 1960, Sheldon Glashow succeeded in unifying electromagnetic and weak interactions into a single electroweak theory~\cite{Weinberg:1967tq}, later working concurrently with Steven Weinberg and Abdus Salam to explain the weak boson masses by incorporating the Higgs mechanism~\cite{PhysRevLett.13.508,PhysRevLett.13.321,PhysRevLett.13.585}.  By the mid-seventies, the modern theory of the strong interaction was also completed.  In the decades since, the Standard Model has been strikingly successful as new experiments have verified Standard Model predictions to ever-increasing accuracy.

The Standard Model rests on the concept of quantized energy and momentum relations, forming a set of quantum field theories associated with the fundamental forces.  The properties of the forces are reflected in the symmetries of their respective field theories.  In the language of group theory, the Standard Model can be described as:
\begin{equation}
  SU(3)_C \times SU(2)_\text{L} \times U(1)_Y,
\end{equation}
with the $SU(3)_C$ group corresponding to the strong force generated by color charge $C$, the $SU(2)_\text{L}$ group corresponding to weak isospin $T_3$ (relevant only for left-handed particles), and the $U(1)$ group corresponding to weak hypercharge $Y$.  Each group implies a gauge symmetry which enforces conservation of the associated charge and determines the properties of the resulting gauge bosons which mediate the interaction.  The $SU(2)_\text{L} \times U(1)_Y$ piece describes the mixing and unification of the weak and electromagnetic forces in Glashow's original electroweak theory.  Of particular interest is the non-Abelian nature of this symmetry which gives rise to weak bosons which themselves carry weak charge.  As a result, it becomes possible to have direct interactions between these bosons, with significant implications which will be discussed in Chapter~\ref{chapter:wz-theory}.


\section{Fundamental Particles}
The particle content of the Standard Model is naturally split into \emph{fermions} which constitute matter and \emph{bosons} which carry forces.  The fermions can be further divided into two distinct families --- the \emph{quarks} which interact via the strong nuclear force and the \emph{leptons} which do not (Table~\ref{tab:fermions}).  Among the bosons, the electromagnetic force is mediated by the photon~($\gamma$), the weak force is mediated by the $W$ and the~$Z$, and the strong force is mediated by gluons~($g$) (Table~\ref{tab:bosons}).

\newcommand{\masshead}{\ensuremath{mc^2 / \eV}}

\begin{table*}
  \centering
  \newcommand{\mysep}{$\:\times$ 10}
  \begin{tabular}{r r r r l r@{\mysep}l l r@{\mysep}l l r@{\mysep}l}
    \toprule
    & & & & \multicolumn{3}{c}{$1^\text{st}$ Generation} & \multicolumn{3}{c}{$2^\text{nd}$ Generation} & \multicolumn{3}{c}{$3^\text{rd}$ Generation} \\ 
    \cmidrule(r){4-6} \cmidrule(rl){7-9} \cmidrule(l){10-12}
    $s$ & $Q$ & $T_3$ & $C$ & $f$ &  \multicolumn{2}{c}{\masshead} & $f$ & \multicolumn{2}{c}{\masshead} & $f$ & \multicolumn{2}{c}{\masshead} \\
    \midrule
    $\frac{1}{2}$ & $+\frac{2}{3}$ & $+\frac{1}{2}$ & 1 & $u$ & 3&$^6$ & $c$ & 1.27&$^9$ & $t$ & 1.72&$^{11}$ \\
    $\frac{1}{2}$ & $-\frac{1}{3}$ & $-\frac{1}{2}$ & 1 & $d$ & 5&$^6$ & $s$ & 1.01&$^8$ & $b$ & 4.67&$^9$\\
    $\frac{1}{2}$ & $0$ & $+\frac{1}{2}$ & 0 & $\nu_e$ & < 2.2&$^{0}$ & $\nu_\mu$ & < 1.7&$^{5}$ & $\nu_\tau$ & < 1.55&$^{7}$ \\
    $\frac{1}{2}$ & $-1$ & $-\frac{1}{2}$ & 0 & $e$ & 5.11&$^{5}$ & $\mu$ & 1.06&$^8$ & $\tau$ & 1.78&$^9$ \\
    \bottomrule
  \end{tabular}
  \caption[Properties of the fundamental fermions]{The fundamental fermions, with spin $s$, electric charge $Q$, weak isospin $T_3$ (equal to zero for right-handed particles), presence or absence of color indicated by $C$ (the charge of the strong force, with quarks carrying one unit of red, green, or blue color), and mass $m$.  The common symbol used for each fermion is given by $f$, with up-type quarks in the first row, down-type quarks in the second, neutral leptons (neutrinos) in the third, and charged leptons in the fourth.  For each listed particle, there is a corresponding antiparticle with the same mass, but opposite values of $Q$ and $T_3$.}
  \label{tab:fermions}
\end{table*}

\begin{table*}
  \centering
  \newcommand{\mysep}{$\:\times$ 10}
  \begin{tabular}{c c r r r r c}
    \toprule
    Symbol & Interaction & $s$ & $Q$ & $T_3$ & C & \masshead \\
    \midrule
    $\gamma$ & electromagnetism & 1 & 0 & 0 & 0 &0 \\
    $W$ & weak nuclear force    & 1 & $\pm$ 1 & $\pm$ 1 & 0 & \phantom{>} \num{8.04e10} \\
    $Z$ & weak nuclear force    & 1 &       0 &       0 & 0 & \phantom{>} \num{9.12e10} \\
    $g$ & strong nuclear force  & 1 &       0 &       0 & 1 & 0 \\
    $H$ & ---                   & 0 & 0 & 0 & 0 & > \num{1.15e11}\\
    \bottomrule
  \end{tabular}
  \caption[Properties of the fundamental bosons]{The fundamental bosons with spin $s$, electric charge $Q$, weak isospin $T_3$ (the charge of the $SU(2)$ interaction), presence or absence of color indicated by $C$ (the charge of the strong force, with all bosons colorless except for the bicolored gluon), and mass $m$.}
  \label{tab:bosons}
\end{table*}

The normal matter of everyday life is made up of just three fundamental fermions.  The protons and neutrons that form the nucleus of any atom are each colorless clusters of three quarks, tightly bound together via the strong force.  The proton consists of two up quarks and a down quark ($uud$) while the neutron has one up and two down ($udd$).  No atom would be complete, however, without electrons~($e^{-}$) orbiting the nucleus to balance the positive electric charge of the protons.  Together with the electron neutrino~($\nu_e$, a nearly massless particle which interacts very rarely), these form the first generation of matter particles.  

While the additional two fermion generations (the charm, strange, top, and bottom quarks along with the muon and tau, and their associated neutrinos) are otherwise identical to the first, they carry substantially greater masses.  Due to couplings with the $W$ boson, these heavy fermions are able to participate in interactions which cross generational boundaries and are thus unstable.  They can exist only momentarily before they decay, leaving behind first generation fermions.  Experimental evidence confirms that only these three generations exist~\cite{:2005ema}, but a theoretical explanation for that number is one of the unanswered questions of the Standard Model (discussed further in Sec.~\ref{sec:shortcomings}).

\section{Fundamental Particles in the Context of a Collider}
While the Standard Model provides a pleasantly polished roster of distinct particles, most of these are highly unstable and thus rarely found in nature.  We must use colliders to produce bursts of energy intense enough to produce them, then view the lower mass products into which they decay.  As such, we now take a more pragmatic look at how fundamental particles behave in that context.

The structure of the strong interaction ensures asymptotic freedom~\cite{PhysRevLett.30.1343}, meaning that the strength of the interaction becomes arbitrarily weak only at small separations between quarks; the strength of the interaction actually grows as colored particles move apart.  As a result, quarks simply cannot survive outside the confines of a colorless hadron.  A high-momentum quark immediately begins shedding its energy by pulling $q\bar{q}$ pairs out of the vacuum, thereby providing new partners with which to form colorless bound states.  An experimentalist sees the result of this \emph{hadronization} process as a collimated \emph{jet} of charged and neutral particles.  The total energy of a jet, which is closely related to that of the original quark, can be determined by measuring the momenta of the charged hadrons as they bend in a magnetic field along with the total energy deposited by the charged and neutral particles as they interact with a dense material.  These strong and electromagnetic interactions produce showers of secondary particles which can be directly detected, as described below.  Of the six quark flavors, the notable exception to this rule of hadronization is the extremely massive top quark whose lifetime is too short to form bound states; it instead decays directly to the lighter fermions.

Among the charged leptons, only the electron is truly stable, although its low mass makes it prone to \brem when passing through matter, a process of energy emission in the form of photons which can further split to form new electron-positron pairs.  Assuming a high momentum for the original electron, this splitting is likely to continue several times over, forming a cascade known as an electromagnetic shower.  Despite the splitting, the energy of a primary electron can be determined with high accuracy by measuring the total energy released in the shower.

Surprisingly, the unstable muon often turns out to be a cleaner object for experimental observation than the stable electron, as its high mass suppresses \brem losses.  At \GeV energy scales, its relatively long lifetime (\SI{2.2}{\micro s}) allows a muon to travel through hundreds of meters of matter before decaying to an electron.  The muon, then, can be directly detected by sampling its trajectory as it moves through a magnetic field.

In contrast to the electron and muon, the tau lepton decays much too quickly to be identified directly in a detector.  Reconstruction of a tau relies on piecing together its decay products, which will be some combination of electrons, muons, and jets.

The neutral leptons (\emph{neutrinos}) are the most elusive of the fundamental particles.  They are light enough to be stable and they interact only via the weak force, giving them the unique ability to pass through large quantities of matter undisturbed.  Experiments which detect neutrinos are able to sample only a small fraction of the neutrinos passing through their detectors, so they rely on dedicated sources with large statistics to mitigate the low interaction probability.  In the context of a collider, an individual neutrino is entirely untraceable.  When searching for processes involving a neutrino, the collider experimentalist's only recourse is to employ conservation laws, knowing that no particle in the initial state has a momentum component transverse to the beampipe.  By analyzing the distribution of energy deposits for all detectable particles, we can detect an imbalance in the transverse direction (\MET) to associate with a supposed neutrino (a more detailed description of this technique is given in Sec.~\ref{sec:met}).

Each of the bosons can be observed using some combination of the techniques already discussed.  Photons are stable and can be detected by the same electromagnetic showers seen for electrons.  Gluons ejected from collisions hadronize similarly to the quarks, so they can also be observed through jets of charged particles.  The $W$ and $Z$ bosons may decay through a variety of channels, producing either leptons or hadrons.

\section{Electroweak Symmetry Breaking}

Mathematically, the electromagnetic and weak nuclear forces are nearly identical, suggesting a strong symmetry.  Each can be described by a potential:
\begin{equation}
  V \approx \frac{1}{r}e^{-mr},
\end{equation}
with $r$ the distance between two interacting particles and $m$ the mass of the boson mediating the interaction.  Within quantum field theory, the interaction is described as a set of probabilities proportional to:
\begin{equation}
  \frac{g^4}{(q^2c^2 - m^2c^4)^2},
\end{equation}
with $m$ as before, $q$ the momentum transferred between the interacting particles, and $g$ the coupling associated with the force.  The electromagnetic and weak couplings are intimately related, with $g_\text{EM} = -e$ and $g_\text{weak} = -e \cot{\theta_\text{W}}$ differing only by a multiplicative constant near unity ($\cot(\theta_\text{W}) \approx 1.7$) defined by the Weinberg or ``weak mixing'' angle $\theta_\text{W}$.  The substantial low-energy asymmetry between the weak and electromagnetic forces, then, is not due to the coupling but rather due to the high mass of the $W$ and $Z$ bosons which limits the range of weak interactions in comparison to electromagnetic interactions mediated by the massless photon.

The coupling ``constant'' for an interaction is only an approximation, as its value actually depends on the momentum transfer involved in an interaction.  In contexts where this variation in energy scale becomes a noticeable effect, we speak of a ``running'' coupling constant.  In practice, however, we are indeed able to treat the electromagnetic and weak couplings as constant since the low-energy value of order \num{e-2} increases by only 10\% at the energy scale of $W$ and $Z$ bosons.  At energies much higher than those achievable with current colliders, quantum electrodynamics (QED) predicts that the running of the electromagnetic coupling does eventually become significant, yielding infinite contributions at finite energies which threaten to spoil the theory.  This divergence, though, is generally accepted as an indication that the theory is only a low-energy approximation of some more general interaction, so the true behavior of the coupling at high energy is unknown.  In contrast, the running of the strong force coupling is most pronounced at low energies, so calculations in the theory of quantum chromodynamics (QCD) must always be performed in reference to a particular energy scale.

At high energies ($q/c >> m$), the mass of the mediating boson no longer plays a significant role.  The values of the weak and electromagnetic couplings also converge in this region, leading the two interactions to have comparable strength and thus realizing the unification which is the hallmark of electroweak theory.  Understanding the symmetry within the theory, we now turn our attention to how an element can be introduced which breaks that symmetry in order to accomodate the observed behavior at low energy.

Within the Standard Model, electroweak symmetry breaking is effected through the Higgs mechanism which introduces a Higgs field $\varphi$ which generates mass-like terms in the Lagrangian~\cite{PhysRevLett.13.508,PhysRevLett.13.321,PhysRevLett.13.585}.  The Higgs field is a doublet in the $SU(2)$ electroweak interaction, but a singlet in the $SU(3)$ color interaction,
\begin{equation}
  \varphi = \left(\begin{array}{c}\varphi^+ \\ \varphi^0\end{array}\right).
\end{equation}

This field carries a potential,
\begin{equation}
  V(\varphi) = \mu^2\varphi^\dagger\varphi + \lambda(\varphi^\dagger\varphi)^2,
\end{equation}
with mass parameter $\mu^2$ and Higgs field self-interaction strength $\lambda$.  A positive or null value of $\mu^2$ would mean no Higgs interaction whatsoever; to provide an opportunity for the desired spontaneous symmetry breaking, we choose $\mu^2 < 0$, leading to a potential with degenerate minima,
\begin{equation}
  \varphi^\dagger\varphi = -\frac{\mu^2}{2\lambda} = \frac{v^2}{2},
\end{equation}
with $v=\sqrt{-\mu^2/\lambda}$ the vacuum expectation value of~$\varphi$.

\begin{figure}
\centering
\includegraphics[clip=true,trim=0.3in 0 0 0]{matplotlib/other/higgs}
\caption[The Higgs potential]{A qualitative plot of the Higgs potential ($V(\varphi)$), showing the ``champagne bottle'' shape.  There is no single minimum, but rather a circle of degenerate minima along the base of the bottle.}
\label{higgs-potential}
\end{figure}

This non-zero value for~$\varphi$ allows for interactions of massless free particles with the Higgs field at all points in space, making them appear massive.  In particular, the Standard Model Lagrangian includes terms quadratic in the fields for the vector bosons, leading to masses given by:
\begin{align}
  M(W) &= \frac{v|g|}{2}, \\
  M(Z) &= \frac{v\sqrt{g^2 + g^{\prime 2}}}{2}.
\end{align}
with $g$ and $g^\prime$ the couplings associated with $SU(2)$ and $U(1)$ gauge groups respectively.  At this point, we have achieved the electroweak symmetry breaking which was the original intention of the Higgs mechanism, although the theory can be extended to generate masses for the fermions as well.  If we choose to re-express the theory in terms of the field:
\begin{equation}
  \tilde{\varphi} = \varphi - \varphi_0,
\end{equation}
with $\varphi_0$ the Higgs field, we end up with ``Yukawa interaction'' terms $g\varphi_0\bar{\psi}\psi$ which correspond to a fermion with mass $g\varphi_0$.  At present, we have no theoretical motivation for the values of these Yukawa couplings $g$, leading to another set of parameters which must be experimentally derived.

While this is the simplest proposed mechanism for imparting mass to the Standard Model particles, we have yet to observe a Higgs boson, and discovering a Higgs is indeed one of the major physics goals of the LHC.  

\section{Shortcomings of the Standard Model}
\label{sec:shortcomings}

While the proposed Higgs mechanism in the Standard Model provides a comparatively simple explanation for electroweak symmetry breaking, it leaves open a variety of theoretical questions.  In particular, this elementary Higgs model \cite{Lane:2000pa,Shrock:2007km}
\begin{itemize}
\item provides no \emph{dynamical} explanation for electroweak symmetry breaking in the sense that the vacuum expectation value of the Higgs must be experimentally derived ($\mu^2$ could just as well be positive or zero, spoiling the theory),
\item seems \emph{unnatural} since it requires an enormously precise fine-tuning of parameters to avoid quadratically divergent contributions to the Higgs mass,
\item cannot explain the \emph{hierarchy problem} of a vast gap between the electroweak scale ($\mathcal{O}(\sienergy{e2})$) at which the Higgs gives mass to the weak bosons and the Planck scale ($\mathcal{O}(\sienergy{e19})$) at which gravity is expected to have a similar strength the SM forces,
\item is \emph{trivial} in that it is understood to be invalid beyond some cutoff scale $\Lambda$, and
\item provides no insight into \emph{flavor physics}, giving no explanation for fermion generations, masses, or mixing.
\end{itemize}

The triviality problem refers to the same behavior already discussed in quantum electrodynamics which predicts divergent contributions at high energy.  The problem is so named because the only way to avoid the divergent catastrophe without adding new elements to the theory is to require that the normalized charge be zero, leading to a ``trivial'' theory of noninteracting particles.  This characteristic is not generally seen as a problem in QED since the energy scale at which the theory becomes inconsistent is inaccessibly large, suggesting that the theory is a successful low-energy approximation of some more fundamental set of interactions.  The luxury of ignoring divergences, however, cannot be indulged for the Higgs mechanism as the predicted cut-off scale is much lower, perhaps within the energy reach of the LHC.

Several of these issues (particularly unnaturalness and the hierarchy problem) can be mitigated in supersymmetric models~\cite{Lane:2000pa}; indeed, LHC Higgs searches typically consider various supersymmetric configurations alongside the SM Higgs.  This thesis does not consider supersymmetry, but does consider various Higgless models (see Section~\ref{sec:technicolor}) which can also overcome these difficulties.

The success of Glashow's electroweak theory in unifying the electromagnetic and weak forces seems to suggest that all the fundamental forces may really be different aspects of one unified force, but the SM fails to fully integrate the strong force with the electroweak interaction and ignores gravity completely.  All efforts thus far to develop a quantum theory of gravity have failed, as quantum models seem incompatible with general relativity.  For the strong force, there is more hope, and a variety of so-called Grand Unified Theories have been proposed to fold color in with the electroweak interaction (see section \ref{sec:wprime}).

Other problems with the SM involve its limited scope.  While the SM has provided some tremendously accurate predictions, it relies on an unreasonable number of \emph{ad hoc} parameters which must be experimentally derived, including all the particle masses and couplings.  Besides this, the past few decades have produced several astronomical observations inconsistent with the existing content of the model.  In some regions of space where gravitational effects indicate matter should be present, we observe none of the radiation expected from the known massive particles, prompting speculation on new neutrino-like \emph{dark matter} candidates with no electromagnetic or strong couplings, but with mass great enough to explain the observed gravitational effects.  We have also observed an overall outward acceleration of the universe incompatible with any known force; the most promising explanations for this are \emph{dark energy} models where some new quantum field acquires a vacuum expectation value, but we have little to guide as at this point as to the details of such a theory.  Finally, the Standard Model fails to provide any mechanism which could explain the substantial dominance of matter over antimatter in the universe; while several experiments have confirmed some deviation in the behavior of matter vs.\ antimatter with respect to weak interactions, the small magnitude of the effect fails to provide any compelling explanation for the complete absence of bulk antimatter.

% Why a fundamental unit of charge?
% not enough CP violation

\resetlinenumber
\chapter{A Theoretical View of~Diboson~Production}
\label{chapter:wz-theory}

\section{Electroweak Diboson Production}

While all electroweak interactions involve at least one of the bosons $\gamma$, $W$, or $Z$, we can gain new insight into the structure of electroweak theory by considering interactions involving multiple electroweak bosons.  These interactions occur less frequently than single-boson events, but they lie within reach for modern colliders.  Indeed, all triple-boson couplings predicted to occur within the SM have already been observed (see discussion in Chapter~\ref{chap:previous}).

In order to participate in a given interaction, a particle must have a non-zero coupling to the associated boson, corresponding to a non-zero charge.  Table~\ref{tab:bosons}, describing the properties of the various gauge bosons, lists a non-zero value of weak isospin $T_3$ (corresponding to the $SU(2)$ interaction) only for the $W$ while the weak hypercharge $Y = 2(Q - T_3)$ associated with the $U(1)$ interaction is null for all gauge bosons.  As a result, the only couplings allowed in the SM directly between the various electroweak bosons is through the weak isospin of the $W$ which connects it to both the photon and the $Z$.  Thus, we expect to see $WWZ$ and $WW\gamma$ vertices, but never $ZZZ$, $ZZ\gamma$, $Z\gamma\gamma$, or $\gamma\gamma\gamma$; other conceivable combinations are forbidden because they would not conserve electric charge.

The values of the various charges ascribed to the electroweak bosons can be understood in terms of the gauge structure of the two interactions involved.  The observed neutral bosons $Z$ and $\gamma$ are in fact superpositions of the neutral $SU(2)$ boson $W^0$ and the $U(1)$ boson $B$.  Terms in the Lagrangian corresponding to multi-boson interactions arise from non-zero commutation relations within the corresponding group.  Because operators from different groups commute and because each operator necessarily commutes with itself, we cannot build any non-zero term involving only $W^0$ and $B$ operators.  The $WWZ$ and $WW\gamma$ interactions arise from terms which invoke the non-zero commutation relations between $W^0$ and $W^\pm$.

The simplest diagrams leading to diboson production can be drawn through simple reconfiguration of the familiar vertices which couple the gauge bosons to fermion pairs; at a hadron collider, this takes the form of two quarks individually radiating gauge bosons in the same event (Fig.~\ref{fig:dibosons-radiation}).  
%Because it carries an electric charge, the $W$ boson can also interact directly with the photon.  Furthermore, the non-Abelian nature of the weak interaction means that the weak bosons themselves carry weak charge and can interact directly.  This leads to the trilinear gauge couplings (TGCs) $WW\gamma$ and $WWZ$ as well as the quadrilinear gauge couplings (QGCs) .   
The annihilation of fermions to a single gauge boson with subsequent radiation of an additional boson (Fig.~\ref{fig:dibosons-annihilation}) involves the previously mentioned trilinear couplings while additional quartic couplings (QGCs) $W^+W^-W^+W^-$, $W^+W^-Z^0Z^0$, $W^+W^-\gamma^0\gamma^0$, and $W^+W^-Z^0\gamma^0$ come into play in diboson scattering events (Fig.~\ref{fig:dibosons-quartic}).  Finally, the SM predicts diagrams involving a Higgs boson which can decay to gauge boson pairs (Fig.~\ref{fig:dibosons-higgs}).

\begin{figure}
  \centering
  \begin{tikzpicture}[thin, level/.style = {level distance = 1.6cm, line width = 1pt} ]
  \coordinate 
  child[grow = 200] { 
    edge from parent [fermion, backwards] node [below = 2pt] {$q$}
  }
  child[grow = north, level distance = 1.0cm] { 
    child[grow = 160] {
      edge from parent [fermion, forwards] node [above = 2pt] {$q$}}
    child[grow = 30] { 
      edge from parent [boson] node [above = 2pt] {$\gamma$}}
    edge from parent [fermion, forwards] node [left = 0pt] {$q$}
  }
  child[grow = -30] {
    edge from parent [boson] node [below = 2pt] {$\gamma$}
  }
  ;
\end{tikzpicture}

  \hspace{1in}
  \begin{tikzpicture}[thin, level/.style = {level distance = 1.6cm, line width = 1pt} ]
  \coordinate 
  child[grow = 200] { 
    edge from parent [fermion, backwards] node [below = 2pt] {$q$}
  }
  child[grow = north, level distance = 1.0cm] { 
    child[grow = 160] {
      edge from parent [fermion, forwards] node [above = 2pt] {$q$}}
    child[grow = 30] { 
      edge from parent [boson] node [above = 2pt] {$W$}}
    edge from parent [fermion, forwards] node [left = 0pt] {$q^\prime$}
  }
  child[grow = -30] {
    edge from parent [boson] node [below = 2pt] {$W$}
  }
  ;
\end{tikzpicture}

  \caption{Example diagrams of diboson production through radiation from quarks.}
  \label{fig:dibosons-radiation}
\end{figure}

\begin{figure}
  \centering
  \begin{tikzpicture}[thin, level/.style = {level distance = 1.6cm, line width = 1pt} ]
  \coordinate 
  child[grow = 150] { 
    edge from parent [fermion, backwards] node [above = 2pt] {$q$} 
  }
  child[grow = -150] { 
    edge from parent [fermion, forwards] node [below = 2pt] {$q^\prime$} 
  }
  child[grow = east] { 
    child[grow = 30] {
      edge from parent [boson] node [above = 2pt] {$W$}}
    child[grow = -30] { 
      edge from parent [boson] node [below = 2pt] {$\gamma$}}
    edge from parent [boson] node [above = 0pt] {$W$}
  }
  ;
\end{tikzpicture}

  \caption{Example diagram of diboson scattering through quark annihilation.}
  \label{fig:dibosons-annihilation}
\end{figure}

\begin{figure}
  \centering
  \begin{tikzpicture}[thin, level/.style = {level distance = 1.6cm, line width = 1pt} ]
  \coordinate 
  child[grow = 200] { 
    edge from parent [fermion, backwards] node [below = 2pt] {$q$}
  }
  child[grow = north] { 
    child[grow = 160] {
      edge from parent [fermion, forwards] node [above = 2pt] {$q$}
    }
    child[grow = -30] { 
      child[grow = 30] {
        edge from parent [boson] node [above = 2pt] {$W$}}
      edge from parent [boson] node [above = 2pt] {$W$}
    }
    edge from parent [fermion, forwards] node [left = 0pt] {$q^\prime$}
  }
  child[grow = 30] {
    child[grow = -30] {
      edge from parent [boson] node [below = 2pt] {$W$}}
    edge from parent [boson] node [below = 2pt] {$W$}
  }
  ;
\end{tikzpicture}  

  \caption{Example diagram of diboson scattering involving a quartic gauge coupling.}
  \label{fig:dibosons-quartic}
\end{figure}

\begin{figure}
  \centering
  \begin{tikzpicture}[thin, level/.style = {level distance = 1.6cm, line width = 1pt} ]
  \coordinate 
  child[grow = west] { 
    % bottom incoming quark
    edge from parent [fermion, backwards] node [below = 2pt] {$q$}
  }
  child[grow = 30] { 
    child[grow = 150] {
      child[grow = west] {
        % top incoming quark
        edge from parent [fermion, forwards] node [above = 2pt] {$q$}
      }
      child[grow = 30] {
        edge from parent [fermion, backwards] node [left=2pt, above=0pt] {$q^\prime$}
      }
      edge from parent [boson] node [above=4pt, right=0pt] {$W$}
    }
    child[grow = east] {
      child[grow = 30] {
        edge from parent [boson] node [left = 2pt, above = 2pt] {$W$}
      }
      child[grow = -30] {
        edge from parent [boson] node [left = 2pt, below = 2pt] {$W$}
      }
      edge from parent [fermion, forwards] node [below = 0pt] {$H$}
    }
    edge from parent [boson] node [below=4pt, right=0pt] {$W$}
  }
  child[grow = -30] {
    edge from parent [fermion, forwards] node [left = 2pt, below=0pt] {$q^\prime$}
  }
  ;
\end{tikzpicture}  

  \caption{Example diagram of diboson production involving a Higgs boson.}
  \label{fig:dibosons-higgs}
\end{figure}

Each diagram given in Figs.~\ref{fig:dibosons-radiation} through~\ref{fig:dibosons-higgs} shows the simplest configuration which leads to that interaction.  Experimental measurements, however, cannot discriminate between these diagrams and more complex ones which yield the same final state.  In general, the contribution from a given diagram decreases rapidly as the number of vertices increases, since each vertex introduces a suppression to the interaction probability on the order of the coupling value, meaning that higher order diagrams can be ignored.  The same is not necessarily true in the case of QCD interactions where the coupling can be of order unity.  We are fortunate that at the energy scale of weak bosons, the coupling is small enough that a ``perturbative QCD''~cite{Ellis1979285} approach which considers only some finite set of the simplest diagrams can provide the needed precision.  It becomes useful then to talk about the maximum ``order'' in the QCD coupling $\alpha_\text{s}$ considered for a given calculation.  The simplest diagrams are ``leading order'' while those involving one or two extra factors of $\alpha_\text{s}$ are ``next-to-leading'' (NLO) or ``next-to-next-to-leading'' (NNLO).

\section{Associated $WZ$ Production}

\begin{figure*}
  \center
  \newcommand{\spacer}{\hspace{0.5in}}
  \subbottom{
    \begin{tikzpicture}[thin, level/.style = {level distance = 2.0cm, line width = 1pt} ]
      \coordinate 
      child[grow =  90] {edge from parent [boson] node [right = 0pt] {$Z$}}
      child[grow = 210] {edge from parent [boson] node [above = 1pt] {$W$}}
      child[grow = 330] {edge from parent [boson] node [above = 1pt] {$W$}}
      ;
    \end{tikzpicture}
  }
  \spacer
  \subbottom{
    \begin{tikzpicture}[thin, level/.style = {level distance = 2.0cm, line width = 1pt} ]
      \coordinate 
      child[grow =  45] {edge from parent [boson] node [right = 4pt] {$W$}}
      child[grow = 135] {edge from parent [boson] node [left  = 4pt] {$W$}}
      child[grow = 225] {edge from parent [boson] node [left  = 4pt] {$W$}}
      child[grow = 315] {edge from parent [boson] node [right = 4pt] {$W$}}
      ;
    \end{tikzpicture}
  }
  \spacer
  \subbottom{
    \begin{tikzpicture}[thin, level/.style = {level distance = 2.0cm, line width = 1pt} ]
      \coordinate 
      child[grow =  45] {edge from parent [boson] node [right = 4pt] {$Z$}}
      child[grow = 135] {edge from parent [boson] node [left  = 4pt] {$W$}}
      child[grow = 225] {edge from parent [boson] node [left  = 4pt] {$W$}}
      child[grow = 315] {edge from parent [boson] node [right = 4pt] {$Z$}}
      ;
    \end{tikzpicture}
  }
\caption{The three vertices giving direct couplings between the weak bosons.}
\label{fig:tgc-vertices}
\end{figure*}

The particular focus of this thesis is on events where $W$ and $Z$ bosons are produced in association from the same hard-scattering interaction.  Within the SM, there are two QGCs and one TGC which involve both the $W$ and the $Z$ (Fig.~\ref{fig:tgc-vertices}).  The two leading order diagrams (given in Fig.~\ref{fig:wz-diagrams}) which contribute to $WZ$ production at the LHC are the $t$-channel process whereby a quark and antiquark emit $W$ and $Z$ bosons through exchange of a quark propagator and the $s$-channel process in which two quarks annihilate to an off-shell $W$ with subsequent radiation of a $Z$ boson.  There exist many possibilities for the subsequent decays of the vector bosons (Fig.~\ref{pie-decays}), but the cleanest experimental signatures come from their leptonic decays.

\begin{figure*}[!htbp]
  \centering
  \subbottom[$s$-channel]{
    \label{fig:wz-schannel}
    \begin{tikzpicture}[thin, level/.style = {level distance = 1.6cm, line width = 1pt} ]
  \coordinate 
  child[grow = 150] { 
    edge from parent [fermion, backwards] node [above = 2pt] {$q$} 
  }
  child[grow = -150] { 
    edge from parent [fermion, forwards] node [below = 2pt] {$q^\prime$} 
  }
  child[grow = east] { 
    child[grow = 30] {
      child[grow = 15] {
        edge from parent [fermion, forwards] node [above = 0pt] {$\ell$}}
      child[grow = -15] {
        edge from parent [fermion, backwards] node [below = 0pt] {$\nu_\ell$}}
      edge from parent [boson] node [above = 2pt] {$W$}}
    child[grow = -30] { 
      child[grow = 15] {
        edge from parent [fermion, forwards] node [above = 0pt] {$\ell^\prime$}}
      child[grow = -15] {
        edge from parent [fermion, backwards] node [below = 0pt] {$\ell^\prime$}}
      edge from parent [boson] node [below = 2pt] {$Z$}}
    edge from parent [boson] node [above = 0pt] {$W$}
  }
  ;
\end{tikzpicture}

  }
  \hspace{0.8in}
  \subbottom[$t$-channel]{
    \label{fig:wz-tchannel}
    \begin{tikzpicture}[thin, level/.style = {level distance = 1.6cm, line width = 1pt} ]
  \coordinate 
  child[grow = 200] { 
    edge from parent [fermion, backwards] node [below = 2pt] {$q$}
  }
  child[grow = north, level distance = 1.0cm] { 
    child[grow = 160] {
      edge from parent [fermion, forwards] node [above = 2pt] {$q^\prime$}}
    child[grow = 30] { 
      child[grow = 15] {
        edge from parent [fermion, forwards] node [above = 0pt] {$\ell$}}
      child[grow = -15] {
        edge from parent [fermion, backwards] node [below = 0pt] {$\nu_\ell$}}
      edge from parent [boson] node [above = 2pt] {$W$}}
    edge from parent [fermion, forwards] node [left = 0pt] {$q$}
  }
  child[grow = -30] {
    child[grow = 15] {
      edge from parent [fermion, forwards] node [above = 0pt] {$\ell^\prime$}}
    child[grow = -15] {
      edge from parent [fermion, backwards] node [below = 0pt] {$\ell^\prime$}}
    edge from parent [boson] node [below = 2pt] {$Z$}
  }
  ;
\end{tikzpicture}

  }\phantom{\hspace{0.2in}}
  \\
  \subbottom[quartic scattering]{
    \label{fig:wz-scattering}
    \begin{tikzpicture}[thin, level/.style = {level distance = 1.6cm, line width = 1pt} ]
  \coordinate 
  child[grow = west] {
    % bottom-left quark
    edge from parent [fermion, backwards] node [below = 2pt] {$q$}
  }
  child[grow = north] { 
    child[grow = west] {
      % top-left quark
      edge from parent [fermion, forwards] node [above = 2pt] {$q^\prime$}
    }
    child[grow = -30] { 
      child[grow = 30] {
        child[grow = 15] {
          edge from parent [fermion, forwards] node [above = 0pt] {$\ell$}
        }
        child[grow = -15] {
          edge from parent [fermion, backwards] node [below = 0pt] {$\nu_\ell$}
        }
        edge from parent [boson] node [above = 2pt] {$W$}
      }
      edge from parent [boson] node [above = 2pt] {$W$}
    }
    edge from parent [fermion, forwards] node [left = 0pt] {$q$}
  }
  child[grow = 30] {
    child[grow = -30] {
      child[grow = 15] {
        edge from parent [fermion, forwards] node [above = 0pt] {$\ell^\prime$}
      }
      child[grow = -15] {
        edge from parent [fermion, backwards] node [below = 0pt] {$\ell^\prime$}
      }
      edge from parent [boson] node [below = 2pt] {$Z$}
    }
    edge from parent [boson] node [below = 2pt] {$Z$}
  }
  ;
\end{tikzpicture}
    
  }
  \hspace{0.5in}
  \subbottom[Higgs-mediated scattering]{
    \label{fig:wz-scattering-higgs}
    \begin{tikzpicture}[thin, level/.style = {level distance = 1.6cm, line width = 1pt} ]
  \coordinate 
  child[grow = west] {% bottom-left quark from qqZ vertex
    edge from parent [fermion, backwards] node [below = 2pt] {$q$}
  }
  child[grow = 30] {% initial Z
    child[grow = north, level distance = 0.7cm] {% Higgs
      child[grow = 150] {% initial W
        child[grow = west] {% top-left incoming quark
          edge from parent [fermion, forwards] node [above = 2pt] {$q^\prime$}
        }
        child[grow = 30] {% top-left incoming quark
          edge from parent [fermion, forwards] node [above = 2pt] {$q$}
        }
        edge from parent [boson] node [above = 5pt, right = 0] {$W$}
      }
      child[grow = 30] {% final W
        child[grow = 15] {
          edge from parent [fermion, forwards] node [above = 0pt] {$\ell$}
        }
        child[grow = -15] {
          edge from parent [fermion, backwards] node [below = 0pt] {$\nu_\ell$}
        }
        edge from parent [boson] node [above = 5pt, left = 0] {$W$}
      }
      edge from parent [fermion] node [right = 0pt] {$H$}
    }
    child[grow = -30] {% final Z
      child[grow = 15] {
        edge from parent [fermion, forwards] node [above = 0pt] {$\ell^\prime$}
      }
      child[grow = -15] {
        edge from parent [fermion, backwards] node [below = 0pt] {$\ell^\prime$}
      }
      edge from parent [boson] node [below = 5pt, left = 0] {$Z$}
    }
    edge from parent [boson] node [below = 5pt, right = 0] {$Z$}
  }
  child[grow = -30] {
    edge from parent [fermion, forwards] node [below = 2pt] {$q$}
  }
  ;
\end{tikzpicture}

  }
  \caption[Diagrams for $WZ$ production]{Major production modes contributing to the $WZ$ states under study.  The leading order $s$-channel and $t$-channel processes dominate.  The quartic scattering diagram, by itself divergent, is balanced in the SM by interference from the Higgs-mediated scattering diagram.  The final-state leptons $\ell$ and~$\ell^{\prime}$ may be either electrons or muons.}
  \label{fig:wz-diagrams}
\end{figure*}

Although much more rare, processes that involve the scattering of longitudinally polarized gauge bosons can also result in the production of $WZ$ pairs (Fig.~\ref{fig:wz-scattering}).  These are particularly interesting because the amplitudes for such scattering processes violate unitarity at the \TeV scale in the absence of an interfering process to suppress the contribution~\cite{Han:2009em}.  The simplest scenarios which can provide such a process involve either a SM Higgs (Fig.~\ref{fig:wz-scattering-higgs}) or some new particle with similar characteristics.  Direct observation of these scattering processes is within the reach of the LHC, but the required collision statistics (on the order of \SI{100}{\fbinv}) for an observation will likely not be available for several years.  Alternative mechanisms for breaking the electroweak symmetry, however, could lead to enhancements which would make this process observable more quickly.  Thus, measurements of associated $WZ$ production provide tantalizing insights into the structure of the electroweak theory regardless of outcome; disagreement with Standard Model predictions would indicate new physics while agreement provides further evidence for the existence of a Higgs particle providing the needed interference.

The Standard Model Lagrangian contains the following terms to describe the $WWZ$ coupling:
\begin{equation}
  \label{eq:sm-tgc-lagrangian}
  \mathcal{L}_{WWZ}^\text{SM} = -i g_{WWZ}\, \left[(W_{\mu\nu}^\dag W^\mu - W^{\mu\dag}W_{\mu\nu})\, Z^\nu + W_\mu^\dag W_\nu\, Z^{\mu\nu} \,\right],
\end{equation}
with $W_\mu$ denoting the $W$ field, $W_{\mu\nu} = \partial W_\nu - \partial W_\mu$, $Z_\mu$ denoting the $Z$ field, $Z_{\mu\nu} = \partial Z_\nu - \partial Z_\mu$, and coupling $g_{WWZ} = -e \cot{\theta_\text{W}}$.  New physics could add extra terms which augment the SM contribution to this vertex.  Such anomalous TGCs provide a model-independent language with which to discuss constraints on physics beyond the Standard Model parameterized in terms of an \emph{effective} Lagrangian~\cite{Hagiwara:1986vm,doi:10.1146/annurev-nucl-102010-130106}:
\begin{align}
  \mathcal{L}_{WWZ}^\text{eff} = -i g_{WWZ}\; \Big[\;
  & g_1^Z\, (W_{\mu\nu}^\dag W^\mu - W^{\mu\dag}W_{\mu\nu})\, Z^\nu + \nonumber\\
  & k_Z\, W_\mu^\dag\, W_\nu\; Z^{\mu\nu} + \\
  & \frac{\lambda_Z}{M_W^2}\, W_\mu^{\nu\dagger}\, W_\nu^\rho\; Z_\rho^\mu\; \Big],\nonumber
\end{align}
which reduces to $\mathcal{L}_{WWZ}^\text{SM}$ by setting $g_1^Z = k_Z = 1$ and $\lambda_Z = 0$.  Searches for new physics through anomalous gauge couplings typically present their results as limits on the deviation of these parameters from their SM values.

%TODO: what specific changes would be expected for different models under consideration?

In a particle experiment, we are often interested in predicting and measuring the rate at which different types of interactions occur.  The observed event rate ($dN/dt$) is highly dependent on the particular configuration of the experiment at any given moment, so we tend to express it as a \emph{cross section} ($\sigma$) which describes the likelihood of the interaction multiplied by a \emph{luminosity} ($\mathcal{L}$) which describes the intensity of the beam,
\begin{equation}
  \label{eq:cross-section}
  \frac{dN}{dt} = \sigma\mathcal{L}.
\end{equation}
The cross section depends only on the energy of the collider, so it serves as a convenient characterization of the probability of a given process occurring.

By integrating Eq.~\ref{eq:cross-section} with respect to some period of collision activity, we obtain an \emph{integrated luminosity}:
\begin{equation}
  L = \int_{t_0}^{t} \mathcal{L}\,dt = \frac{N}{\sigma},
\end{equation}
for the number of produced events $N$ for a process with cross section $\sigma$ over a period $\Delta t = t - t_0$.  The integrated luminosity is a convenient measure of the quantity of collision data produced in an experiment since it has dimensions of inverse cross section, typically expressed in \pbinv or $\fbinv = \SI{1000}{\pbinv}$.  For a process with $\sigma = \SI{10}{pb}$, for example, we would expect on average one event for every \SI{0.1}{\pbinv} of integrated luminosity.

Conceptually, the cross section for a process is analogous to the area presented by a target to a stream of incoming projectiles, but it takes into account that particle interactions are described by probabilities rather than hits and misses.  One goal of this thesis will be to measure the cross section for $WZ$ production at the LHC (Chapter~\ref{chapter:cross-section}).

For LHC operation at $\sqrt{s} = \SI{7}{\TeV}$, we expect~\cite{Campbell:2011bn}
\begin{equation}
  \label{eq:predicted-cross-section}
  \sigma_\text{NLO}(p + p \to W^\pm + Z) = 18.57 \pm 0.95\,\si{pb}
\end{equation}
based on the most up-to-date theoretical predictions.  This next-to-leading order (NLO) calculation takes into account spin correlations as well as corrections for the probability of radiating an additional jet, but the calculated value is still dominated by contributions from the leading order $s$-channel and $t$-channel diagrams.  These calculations must rely on measurable quantities such as charges which have some dependence on the energy of the interaction, necessitating the choice of some \emph{renormalization scale} to obtain a result.  Additionally, these calculations rely on \emph{factorization} of the QCD pieces of the calculation into short-distance interactions among individual partons accompanied by long-distance interactions related to hadron formation~\cite{Tung:2009}.  The choice of scale introduces uncertainty into the calculation.  The $WZ$ cross section prediction above sets both the renormalization and factorization scales at the average weak boson mass ($(M(W) + M(Z)) / 2$), then estimates errors by repeating the calculation with scale variations around that central value.

\begin{figure*}[!htbp]
\centering
\subbottom[$W$ boson decays]{\label{pie-decays-W}\includegraphics{matplotlib/other/pie-decays-W}}
\hspace{0.5in}
\subbottom[$Z$ boson decays]{\label{pie-decays-Z}\includegraphics{matplotlib/other/pie-decays-Z}}
\caption[Branching fractions for the $W$ and $Z$ bosons]{Branching fractions for the $W$ and $Z$.  Only $1.4\%$ of $WZ$ events lead to the desired trilepton final states, but these are by far the easiest decays to detect with in a collision experiment.}
\label{pie-decays}
\end{figure*}

\section{Possibilities for New Physics}
\label{sec:newphysics}

Experimentalists can take one of two approaches to search for evidence of new physics in $WZ$ production.  This thesis focuses on a search for an excess in the $WZ$ invariant mass spectrum.  Any new particle which can decay to $WZ$ would produce such an excess, revealing its mass.  Another approach is to look for anomalous couplings between the weak bosons.  Within the Standard Model, the $SU(2) \times U(1)$ symmetry of the electroweak interaction completely fixes the $WWZ$ coupling~\cite{Baur:1994aj}.  Thus, any deviation in the coupling would necessitate new physics.  This approach is sensitive even to particles which lie beyond the mass reach of the LHC, since new particles with couplings to the $W$ and $Z$ could act as propagators, adding new diagrams in analogy to Fig.~\ref{fig:wz-schannel} and leading to loop corrections for the $WWZ$ vertex.

\subsection{Technicolor}
\label{sec:technicolor}

\begin{figure*}
\centering
\begin{tikzpicture}[thin, level/.style = {level distance = 1.6cm, line width = 1pt} ]
  \coordinate 
  child[grow = 150] { 
    edge from parent [fermion, backwards] node [above = 2pt] {$q$} 
  }
  child[grow = -150] { 
    edge from parent [fermion, forwards] node [below = 2pt] {$q^\prime$} 
  }
  child[grow = east] { 
    child[grow = east] {
      child[grow = 30] {
        child[grow = 15] {
          edge from parent [fermion, forwards] node [above = 0pt] {$\ell$}}
        child[grow = -15] {
          edge from parent [fermion, backwards] node [below = 0pt] {$\nu_\ell$}}
        edge from parent [boson] node [above = 2pt] {$W$}}
      child[grow = -30] { 
        child[grow = 15] {
          edge from parent [fermion, forwards] node [above = 0pt] {$\ell^\prime$}}
        child[grow = -15] {
          edge from parent [fermion, backwards] node [below = 0pt] {$\ell^\prime$}}
        edge from parent [boson] node [below = 2pt] {$Z$}}
      edge from parent [technihadron] node [above = 0pt] {\technirho}
    }
    edge from parent [boson] node [above = 0pt] {$W$}
  }
  ;
\end{tikzpicture}

\caption[Primary diagram for \technirho production]{Production and decay of a \technirho.  The quarks produce an intermediate off-shell $W$ decaying to a pair of techniquarks which form a \technirho bound state with subsequent decay to $WZ$.}
\label{fig:feynman-technirho}
\end{figure*}

Various theories have sought to explain the abundance of distinct particle types currently believed to be fundamental by exploring the possibility that they may actually have substructure.  These compositeness theories have the power to both simplify the particle zoo and explain electroweak symmetry breaking without the need for a Higgs boson.  The most enduring class of compositeness models is Technicolor, which proposes a new interaction modeled on the strong force that can achieve dynamical breakdown of electroweak symmetry~\cite{Weinberg:1979bn,Susskind:1978ms}, eliminating the naturalness, hierarchy, and triviality problems inherent in the SM Higgs~\cite{Lane:2000pa}. Like the strong interaction, Technicolor would feature asymptotic freedom, encouraging the formation of bound states with no net Technicolor charge.

Technicolor in its original form was quickly ruled out because of its prediction of flavor-changing neutral currents which had not been observed experimentally.  However, the more recent Extended Technicolor (ETC) models employ a slowly-running or ``walking'' gauge coupling which allows the theory to generate realistic masses for fermions and to suppress the flavor-changing neutral currents~\cite{Holdom:1981rm}. As an additional consequence of the walking coupling, the predicted masses of the Technicolor particles are lower than previously expected, leading to  a Low-Scale Technicolor (LSTC)~\cite{Eichten:1996dx} spectrum accessible at the LHC.

Technicolor predicts a variety of new bound states of techniquarks, several of which can decay to $WZ$ (see Figure~\ref{fig:feynman-technirho}), making this the most promising channel for an LHC discovery of Technicolor.  Indeed, the presence of new particles coupling to the massive vector bosons is one of the primary features which make the Technicolor idea compelling since such couplings are necessary to provide a viable alternative to the Higgs mechanism.  The availability of new particles which can decay to $WZ$ is especially attractive, as this can provide modifications sufficient to control the $WZ$ scattering divergences above \SI{1}{\TeV}.  In analogy with QCD, the technihadrons with $I^G(J^{PC}) = 1^-(0^{-+})$, $1^+(1^{--})$, and $1^-(0^{++})$ are called $\technipi$, $\technirho$, and $\technia$.  A long-standing problem with walking Technicolor has been a very large value for the precision-electroweak $S$-parameter, a quantity used to provide generic constraints on physics beyond the Standard Model~\cite{Peskin:1990zt}. Recent  models incorporate the idea that the $S$-parameter can be naturally suppressed if the lightest vector technihadron, \technirho, and its axial-vector partner, \technia, are nearly mass degenerate~\cite{Lane:2002sm}.  These technihadrons are expected to have masses below \simass{700}, and their decays have distinctive signatures with narrow resonant peaks.


\subsection{New Heavy Vector Bosons (\wprime)}
\label{sec:wprime}

Many extensions of the Standard Model predict heavy charged vector bosons which can decay to $WZ$~\cite{Salam:1974,Altarelli:1989,Perelstein:2007}.  Such bosons are usually called \wprime, and they can arise due to an extended gauge sector in unification models or due to extra dimensions.

Grand unified theories (GUTs) attempt to yoke the strong and electroweak interactions together under one expanded gauge group ($SU(5)$ in the simplest case)~\cite{Georgi:1974sy}.  In order to fit with our current understanding of the universe, the symmetry of this expanded gauge group must break to give the observed $SU(3) \times SU(2) \times U(1)$ symmetry of the standard model.  This breaking, however, necessarily leaves behind excess symmetries which have yet to be observed.  Any additional $U(1)$ symmetry can be associated with a $Z$-like neutral vector boson while an additional $SU(2)$ symmetry can be associated with a $W$-like charged vector boson.

The greatest argument for GUTs is simply the aesthetic virtue of being able to describe electromagnetic, strong, and weak interactions as different manifestations of a single force, but such theories often carry additional explanatory power.  For example, an expanded gauge symmetry often requires that the charges of the electron and proton be precisely opposite, providing a natural explanation for an otherwise \emph{ad hoc} parameter of the Standard Model.

A \wprime{} boson appears in an entirely different context for models which predict a number of tightly-curled or ``hidden'' extra dimensions~\cite{Klein:1926}.  In these models, the familiar vector bosons can acquire a momentum in one of the additional dimensions, leading to a series of excited states that would appear as more massive versions of the $W$ and $Z$.

Current limits on \wprime searches in leptonic channels are interpreted in the context of the Sequential Standard Model (SSM)~\cite{Salam:1974,Altarelli:1989,Perelstein:2007} and exclude \wprime bosons with masses below \SI{2.27}{\TeV} at 95\% confidence level~\cite{CMS-PAS-EXO-11-024}. While those searches assume that the $\wprime \to W + Z$ decay mode is suppressed, many \wprime models predict a suppression of the coupling to \emph{leptons} instead, leading to a relative enhancement in the triple gauge couplings that could lead to a $WZ$ final state~\cite{Wprimereview}.  For example, there are models in which the \wprime{} couples to new fermions where the decay to new fermion pairs would be suppressed if their masses are larger than the \wprime mass, leading to a dominance of decays into vector bosons~\cite{WprimeWZ}. Therefore, a search for $\wprime \to W + Z$ should be considered complementary to the searches for a \wprime{} decaying directly to leptons.
